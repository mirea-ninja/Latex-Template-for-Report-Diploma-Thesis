%%%%%%%%%%%%%%%%% Оформление ГОСТА%%%%%%%%%%%%%%%%%

% Все параметры указаны в ГОСТЕ на 2021, а именно:

% Шрифт для курсовой Times New Roman, размер – 12 пт.
\setdefaultlanguage[spelling=modern]{russian}
    \setotherlanguage{english}
\defaultfontfeatures{Ligatures={TeX},Renderer=Basic} % задаёт свойства шрифтов по умолчанию
\setmainfont{Times New Roman} 
\setromanfont{Times New Roman} 
\newfontfamily\cyrillicfont{Times New Roman}


% Междустрочный интервал должен быть равен 1.5.
\linespread{1.5} % междустрочный интервал


% Каждая новая строка должна начинаться с отступа равного 1.25 сантиметра.
\setlength{\parindent}{1.25cm} % отступ для абзаца


% Текст, который является основным содержанием, должен быть выровнен по ширине по умолчанию включен из-за типа документа в main.tex

% Ширина левого поля должна равняться 3 сантиметра, а правое 1 сантиметра. Верхнее и нижнее должны равняться 2 сантиметра.
\usepackage[left=3cm,right=1.5cm,top=2cm,bottom=2cm]{geometry} % поля

%%%%%%%%%%%%%%%%%% Дополнения %%%%%%%%%%%%%%%%%%%%%%%%%%%%%%%%%






% Внесение titlepage в учёт счётчика страниц
\makeatletter
\renewenvironment{titlepage} {
	\thispagestyle{empty}
}


% Цвет гиперссылок и цитирования
\usepackage{hyperref} 
 \hypersetup{ 
     colorlinks=true, 
     linkcolor=black, 
     filecolor=blue, 
     citecolor = black,       
     urlcolor=purple, 
     }
     
% Включение в нумерацию рисунка номеров разделов     
\counterwithin{figure}{section}
\counterwithin{figure}{subsection}
\counterwithin{figure}{subsubsection}

