% !TeX program = xelatex
\documentclass[14pt, a4paper, titlepage]{extarticle}

\usepackage[english,main=russian]{babel}
\usepackage{fontspec}
\setmainfont{Times New Roman} % Если возникают проблемы при компиляции с данной строкой, необходимо установить на компьютер Times New Roman 
\usepackage{newtxmath} % Поменять гарнитуру в фомулах на Times New Roman

\usepackage[left=30mm, right=15mm, top=20mm, bottom=20mm]{geometry}

\usepackage{indentfirst} % Красная строка у первого абзаца раздела

\usepackage{graphicx}

\parindent=1.25cm % Размер красной строки

\parskip=0pt % Отступ между абзацами

\righthyphenmin=2 % Разрешить переносить слоги в 2 буквы (стандартное значение 3)

\linespread{1.3} % полуторный межстрочный интервал

\usepackage{tocbibind} % Добавить раздел оглавление в оглавление

% Настройка заголовка оглавления
\addto\captionsrussian{\renewcommand{\contentsname}{Оглавление}}

\usepackage[normalem]{ulem} % underline some lines


\usepackage{tocloft}
% Формат оглавления
\renewcommand\cfttoctitlefont{\hfill\fontsize{16pt}{16pt}\selectfont\bfseries\MakeUppercase}
\renewcommand\cftaftertoctitle{\hfill\hfill}

\renewcommand{\cftsecleader}{\cftdotfill{\cftdotsep}} % Добавить точки у разделов в оглавлении

% Настройка раздела, подраздела, подподраздела
\usepackage{titlesec}
\titleformat{\section}{\parskip=6pt\filcenter\fontsize{16pt}{16pt}\selectfont\bfseries\uppercase}{\thesection}{.5em}{}
\titleformat{\subsection}{\filcenter\bfseries}{\thesubsection}{.5em}{}
\titleformat{\subsubsection}{\filcenter\bfseries}{\thesubsubsection}{.5em}{}

\AddToHook{cmd/section/before}{\clearpage} % Начинать раздел с новой страницы

\renewenvironment{abstract}{\clearpage\section*{\MakeUppercase{\abstractname}}}{\clearpage}

\labelwidth=1.25cm % Горизонтальный отступ у элемента списка

% Ненумерованные списки разной вложенности
\renewcommand\labelitemi{--}
\renewcommand\labelitemii{--}
\renewcommand\labelitemiii{--}
\renewcommand\labelitemiv{--}

% Нумерованные списки разной вложенности
\renewcommand\labelenumi{\arabic{enumi})}
\renewcommand\labelenumii{\asbuk{enumii})}
\renewcommand\labelenumiii{\arabic{enumiii})}
\renewcommand\labelenumiv{\asbuk{enumiv})}

\makeatletter
% Буквы для нумерации списка (исключены ё, з, щ, ч, ъ, ы, ь)
% Подробнее https://ctan.math.illinois.edu/macros/latex/required/babel/contrib/russian/russianb.pdf 
\def\russian@alph#1{\ifcase#1\or
    а\or б\or в\or г\or д\or е\or ж\or
    и\or к\or л\or м\or н\or о\or п\or 
    р\or с\or т\or у\or ф\or х\or ц\or 
    ш\or э\or ю\or я\else\@ctrerr\fi}
\def\russian@Alph#1{\ifcase#1\or
    А\or Б\or В\or Г\or Д\or Е\or Ж\or
    И\or К\or Л\or М\or Н\or О\or П\or 
    Р\or С\or Т\or У\or Ф\or Х\or Ц\or 
    Ш\or Э\or Ю\or Я\else\@ctrerr\fi}

\patchcmd{\l@section}{#1}{\textnormal{\uppercase{#1}}}{}{} % Разделы в оглавлении без выделения жирным, в верхнем регистре
\patchcmd{\l@section}{#2}{\textnormal{#2}}{}{} % Страницы без выделения жирным

\apptocmd{\appendix}{
    \renewcommand{\thesection}{\Asbuk{section}}
    \titleformat{\section}{\filcenter\fontsize{16pt}{16pt}\selectfont\bfseries}{}{0pt}{\MakeUppercase{\appendixname}~\thesection \\}{}{} % Изменение формата раздела приложения
    
    \let\oldsec\section
    \renewcommand{\section}{
        \clearpage
        \phantomsection
        \refstepcounter{section}
        \addcontentsline{toc}{section}{\appendixname~\thesection}
        \oldsec*} % Нумерация раздела после названия
}
\makeatother

\usepackage[labelsep=endash]{caption} % Настройка пунктуации
\captionsetup[table]{justification=raggedright, singlelinecheck=false} % Выравнивание по левому краю надписи таблицы

\addto\captionsrussian{\renewcommand{\figurename}{Рисунок}} % Переопределение caption из babel

% Пакет делает ссылки кликабельными и дает возможность добавить метаданные .pdf документа
\usepackage{hyperref}
% Метаданные .pdf документа, отключение подсветки ссылок
\hypersetup{pdftitle={Методы и средства защиты информации}, pdfauthor={И. О. Фамилия}, colorlinks=false, pdfborder={0 0 0}}

\begin{document}

\begin{titlepage}
    \pagestyle{empty}
    \setlength\parindent{0pt}
    \begin{center}
        \includegraphics[width=2.5cm]{MIREA_Gerb_Black} \par
        МИНОБРНАУКИ РОССИИ \par 
        Федеральное государственное бюджетное образовательное учреждение высшего образования \par
        \textbf{<<МИРЭА --- Российский технологический университет>>} \par
        \textbf{РТУ МИРЭА}
    \end{center}
    \bigskip\par
    \textbf{Институт:} \uline{Институт кибербезопасности и цифровых технологий} \par
    \textbf{Специальность (направление):} \uline{Технологии искусственного интеллекта в безопасности (09.03.02 Информационные системы и технологии)} \par
    \textbf{Кафедра:} \uline{КБ-14 <<Цифровые технологии обработки данных>>} \par
    \textbf{Дисциплина:} \uline{Методы и средства защиты информации} \bigskip\par
    \begin{center}
        ОТЧЕТ ПО \par
        ПРАКТИЧЕСКОЙ РАБОТЕ \textnumero{}~1 \par
        Название работы
    \end{center}
    \begin{flushright}
        Выполнил студент группы \par
        БСБО-05-20: \par
        Фамилия И.О. \par
        Проверил ассистент: \par
        Акимов Э.М.
    \end{flushright}
    \begin{center}
        \vfill \textbf{Москва~\the\year{}~г.}
    \end{center}
\end{titlepage}
\addtocounter{page}{1} % 

\begin{abstract}
    
    \subsection*{Метаданные .pdf}
    
    Не забудьте в преамбуле в команде \verb"\hypersetup" поменять значение полей \verb"pdftitle={"Название моего документа\verb"}", \verb"pdfauthor={"Моё имя\verb"}".
    
\end{abstract}

\tableofcontents

\section*{Введение}
\phantomsection
\addcontentsline{toc}{section}{Введение}

Текст введения.

\section{Структура документа}

Структура документа совместима со стандартным классом документа article.

\subsection{Титульная страница}

Титульная страница находится в окружении titlepage в начале документа:
\begin{verbatim}
\begin{titlepage}
...
\end{titlepage}
\end{verbatim}

При изменении кол-ва титульных страниц необходимо отметить кол-во в счетчике станиц. Пример:
\begin{verbatim}
\begin{titlepage}
...
\end{titlepage}
\addtocounter{page}{1}
\end{verbatim}

\subsection{Аннотация}

\begin{verbatim}
\begin{abstract}
...
\end{abstract}
\end{verbatim}

\subsection{Оглавление}

\begin{verbatim}
\tableofcontents
\end{verbatim}

\subsection{Введение, раздел без нумерации}

\noindent\verb"\section*{"Введение\verb"}"\\
\verb"\phantomsection"\\
\verb"\addcontentsline{toc}{section}{"Введение\verb"}"\\
\verb"..."

\subsection{Раздел}

\begin{verbatim}
\section{...}
...
\end{verbatim}

\subsection{Подраздел}

\begin{verbatim}
\subsection{...}
...
\end{verbatim}

\subsection{Пункт}

\begin{verbatim}
\subsubsection{...}
...
\end{verbatim}

\subsubsection{Пример пункта}

Текст пункта.

\subsection{Приложение}

Как и в стандартных классах перед приложениями необходимо указать команду \verb"\appendix".

Пример с одним приложением:
\begin{verbatim}
\appendix
\section{...}
...
\end{verbatim}

Пример с тремя приложениями:
\begin{verbatim}
\appendix
\section{...}
...
\section{...}
...
\section{...}
...
\end{verbatim}

Ссылка на приложение:
\begin{verbatim}
\ref{...}
\end{verbatim}

Пример ссылки на приложение~\ref{appendix:test_label}.

За порядком приложений необходимо следить самостоятельно.

\subsection{Список}

\begin{verbatim}
\begin{itemize}
    \item ...,
    ...
\end{itemize}
\end{verbatim}

\begin{itemize}
    \item[] Пример списка:
    \item первый уровень вложенности,
    \begin{itemize}
        \item второй,
        \begin{itemize}
            \item третий,
            \begin{itemize}
                \item четвертый.
            \end{itemize}
        \end{itemize}
    \end{itemize}
\end{itemize}

\subsection{Перечисление}

\begin{verbatim}
\begin{enumerate}
    \item ...;
    ...
\end{enumerate} 
\end{verbatim}

\begin{enumerate}
    \item[] Пример перечисления:
    \item первый уровень вложенности;
    \begin{enumerate}
        \item второй;
        \begin{enumerate}
            \item третий;
            \begin{enumerate}
                \item четвертый;
                \item б;
                \item в;
                \item г;
                \item д;
                \item е;
                \item ж;
                \item и;
                \item к;
                \item л;
                \item м;
                \item н;
                \item о;
                \item п;
                \item р;
                \item с;
                \item т;
                \item у;
                \item ф;
                \item x;
                \item ц;
                \item ш;
                \item э;
                \item ю;
                \item я, при наличии б\'ольшего кол-ва элементов компилятор выдаст ошибку.
            \end{enumerate}
        \end{enumerate}
    \end{enumerate}
\end{enumerate}

\subsection{Иллюстрация}
 
Пакет graphicx подключен. См. рисунок~\ref{fig:test_label} на стр.~\pageref{fig:test_label}.

\begin{figure}[htb]
    \centering
    \includegraphics[width=.5\textwidth]{MIREA_Gerb_Black}
    \parskip=6pt
    \caption{Подпись ниже рисунка по центру}
    \label{fig:test_label}
\end{figure}

Обратите внимание, что окружение figure является \emph{плавающим}, и иллюстрация может появиться не там, где Вы ожидаете. Для размещения иллюстрации в конкретное место необходимо воспользоваться опцией H из пакета float (не подключен).

\subsection{Таблица}

См. таблицу~\ref{tab:test_label} на стр.~\pageref{tab:test_label}.

\begin{table}[htb]
    \caption{Подпись выше таблицы слева}
    \centering
    \begin{tabular}{ |c|c|c|c|c| } 
        \hline
        ячейка 1 & ячейка 2 & ячейка 3 & ячейка 4 & ячейка 5 \\ \hline
        ячейка 6 & ячейка 7 & ячейка 8 & ячейка 9 & ячейка 10 \\ \hline
    \end{tabular}
    \label{tab:test_label}
\end{table}

Обратите внимание, что окружение table является \emph{плавающим}, и таблица может появиться не там, где Вы ожидаете. Для размещения таблицы в конкретное место необходимо воспользоваться опцией H из пакета float (не подключен).

\subsection{Уравнение и формула}

См. формулу~\ref{eq:test_label}.

\begin{equation}\label{eq:test_label}
    \text{минус}\,a\times b=c ,
\end{equation}
где $a$ --- первая переменная; \\
$b$ --- вторая переменная; \\
$c$ --- третья переменная.

\appendix

\section{Использованные пакеты}
\label{appendix:used_packages}

\begin{itemize}
    \item caption,
    \item fontspec,
    \item geometry,
    \item graphicx,
    \item hyperref,
    \item indentfirst,
    \item newtxmath,
    \item titlesec,
    \item tocloft,
    \item ulem.
\end{itemize}

\section{Название второго приложения}
\label{appendix:test_label}

Содержание второго приложения.

\end{document}
