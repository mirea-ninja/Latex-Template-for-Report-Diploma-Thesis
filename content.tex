%%%%%%%%%%%%%%%%%%%%%%%%%%%%%%%%%%%%%%%%%%%%%%%%%%%%%%%%%%%%%%%%%%%%%%%%%%%%%%%%%%%%%%%%%%

\section*{Задание}
Написать курсовую/диплом/отчет и разобраться в работе \LaTeX\ .
\addcontentsline{toc}{section}{Задание}

%%%%%%%%%%%%%%%%%%%%%%%%%%%%%%%%%%%%%%%%%%%%%%%%%%%%%%%%%%%%%%%%%%%%
\section{Дерево Иерархии}
Что же такое дерево устройства? Дерево - это набор конфигурационных
файлов, необходимых для определения специфичных для конкретного устройства параметров, пакетов (приложений) и зависимостей. Как правило, дерево устройства состоит из папок \textbf{device, vendor, kernel}.
\begin{itemize}
\item В папке \textbf{device} присутствуют основные конфигурационные файлы, драйвера с открытым кодом.
\item В папке \textbf{vendor} - проприетарные файлы (бинарные файлы с закрытым исходным кодом).
\item В папке \textbf{kernel} – исходники ядра устройства.
\end{itemize}


%%%%%%%%%%%%%%%%%%%%%%%%%%%%%%%%%%%%%%%%%%%%%%%%%%%%%%%%%%%%%%%%%%%%
\section{Основной текст}
Соображения высшего порядка, а также дальнейшее развитие различных форм деятельности требует от нас анализа модели развития. Равным образом выбранный нами инновационный путь обеспечивает широкому кругу специалистов участие в формировании новых предложений?\\
Равным образом рамки и место обучения кадров требует определения и уточнения существующих финансовых и административных условий.
Дорогие друзья, социально-экономическое развитие требует от нас анализа экономической целесообразности принимаемых решений? \cite{kistyakovskii} % здесь \citep используется для вставки цитирования в скобках

\subsection*{Теория 1}

% Можно вручную вставлять рисунки прописывая каждый параметр

\begin{figure}[!htb]
	\centering
	\includegraphics[width=\textwidth]{image1.jpg}
	\caption{Используйте Rich Text}
	\label{fig:image1}
\end{figure}

% Параметра width задаёт ширину рисунка. В этом случае она равна ширине текста, \textwidth. Вместо \textwidth можно укзать значение от 0.1 до 1. \label нужен для того, чтобы потом можно делать сноску на эту картинку, как здесь:


% А можно использовать кастомную команду \image

% \image {Имя изображения.расширение}{Подпись к рисунку}

\image{image2.jpg}{Подпись к рисунку}



\cite{kistyakovskii} показывает, что курс на социально - ориентированный национальный проект способствует повышению актуальности системы масштабного изменения ряда параметров. Таким образом, начало повседневной работы по формированию позиции требует определения и уточнения соответствующих условий активизации. \cite{landau} %обратите внимание, что \citet вставляет цитирование в текст без скобок, чтобы оно вписывалось в текст
показывает, что начало повседневной работы по формированию позиции в значительной степени обуславливает создание дальнейших направлений развития проекта. 

Разнообразный и богатый опыт дальнейшее развитие различных форм деятельности в значительной степени обуславливает создание ключевых компонентов планируемого обновления! Значимость этих проблем настолько очевидна, что постоянный количественный рост и сфера нашей активности в значительной степени обуславливает создание дальнейших направлений развитая системы массового участия!

Практический опыт показывает, что рамки и место обучения кадров играет важную роль в формировании позиций, занимаемых участниками в отношении поставленных задач. Разнообразный и богатый опыт реализация намеченного плана развития обеспечивает широкому кругу специалистов участие в формировании всесторонне сбалансированных нововведений.
Соображения высшего порядка, а также начало повседневной работы по...


% К сожалению таблицы это комплексная вещь, поэтому придется задавать все параметры вручную

\begin{table}[h!]
\centering
\begin{tabular}{|c|c|c|c|} 
 \hline
 Столбец1 & Столбец2 & Столбец3 & Столбец4 \\ [0.5ex] 
 \hline
 \multirow{3}{5em}{Несколько строк} & 6 & 87837 & 787 \\ 
  &  7 & 78 & 5415 \\
   & 545 & 778 & 7507 \\
   & 545 & 18744 & 7560 \\
   & 88 & 788 & 6344 \\ [1ex] 
 \hline
\end{tabular}
\caption{Пример работы с таблицей}
\label{table:1}
\end{table}


Больше о таблицах \href{https://www.overleaf.com/learn/latex/Tables}{тут}.

%%%%%%%%%%%%%%%%%%%%%%%%%%%%%%%%%%%%%%%%%%%%%%%%%%%%%%%%%%%%%%%%%%%%
\section{Математика}

    \subsection{Математические формулы}
    Хорошо известная теорема Пифагора \(x^2 + y^2 = z^2\) была
    доказана недействительной для других показателей.
    Это означает, что следующее уравнение не имеет целочисленных решений:
    \[ x^n + y^n = z^n \]
    Другой способ вставить уравнение в текст такой: $x^2 + y^2 = z^2$. То есть уравнение нужно поместить между двумя знаками "доллара".

    \subsubsection{Дроби}
    При отображении дробей в строке, например \(\frac{3x}{2}\),
    вы можете установить другой стиль отображения:
    \( \displaystyle \frac{3x}{2} \).
    Это также верно и в обратном направлении
    \[ f(x)=\frac{P(x)}{Q(x)} \ \ \textrm{и}
    \ \ f(x)=\textstyle\frac{P(x)}{Q(x)} \]

    \subsubsection{Интегралы}
    Интеграл \(\int_{a}^{b} x^2 dx\) внутри текста.
    \medskip
    Тот же интеграл на дисплее:
    \[
    \int_{a}^{b} x^2 \,dx
    \]
    Официальнный туториал по интегралам можно посмотреть по этой  \href{https://www.overleaf.com/learn/latex/Integrals,_sums_and_limits#Integrals}{ссылке}.

    \subsubsection{Сумма и произведение}
    Тоже оставлю \href{https://www.overleaf.com/learn/latex/Integrals,_sums_and_limits#Sums_and_products}{ссылку}.
    
    \subsubsection{Пределы}
    
    Предел \(\lim_{x\to\infty} f(x)\) внутри текста.
    Тот же предел на дисплее:
    \[
    \lim_{x\to\infty} f(x)
    \]
\begin{figure}[!htb]
	\centering
	\includegraphics[width=\textwidth]{Images/image2.jpg}
	\caption{Пример нумерации рисунков}
	\label{fig:image2}
\end{figure}

%%%%%%%%%%%%%%%%%%%%%%%%%%%%%%%%%%%%%%%%%%%%%%%%%%%%%%%%%%%%%%%%%%%%
\section{Символы}
$\alpha A$ - греческие символы,  $ \lambda; \Lambda$ - физические величины, $\exists; \forall$ - логические символы\\
По этой   \href{https://www.overleaf.com/learn/latex/List_of_Greek_letters_and_math_symbols}{ссылке} можно посмотреть остальные символы. \href{https://www.overleaf.com/learn/latex/Operators}{Здесь} - математические операторы.



%%%%%%%%%%%%%%%%%%%%%%%%%%%%%%%%%%%%%%%%%%%%%%%%%%%%%%%%%%%%%%%%%%%%
\section{Руководство}
\href{https://www.texlive.info/CTAN/info/lshort/russian/lshortru.pdf}{Ссылка} на полное введение в Latex на русском языке.


%%%%%%%%%%%%%%%%%%%%%%%%%%%%%%%%%%%%%%%%%%%%%%%%%%%%%%%%%%%%%%%%%%%%
\newpage
\section{Заключение}
Вот и закончили написание курсовой/диплома/отчета


%%%%%%%%%%%%%%%%%%%%%%%%%%%%%%%%%%%%%%%%%%%%%%%%%%%%%%%%%%%%%%%%%%%%
\begin{center}
  \section*{Вставка кода}
\end{center}
\addcontentsline{toc}{section}{Вставка кода}

%\insertcode{Main.java}{Пример кода}
\subsection*{Файл main.java}


% \codefromfile {Имя файла}
% Для того чтобы файл открывался - нужно поместить его в папку Listings
\codefromfile{Main.java}

\newpage
Можно воспользоваться командой codefromfile, а можно забить код вручную через среду minted:

\subsection*{Файл def.py}
\begin{minted}
[
frame=lines,
framesep=2mm,
baselinestretch=1.2,
fontsize=\footnotesize,
linenos
]
{python}
import numpy as np
    
def incmatrix(genl1,genl2):
    m = len(genl1)
    n = len(genl2)
    M = None #to become the incidence matrix
    VT = np.zeros((n*m,1), int)  #dummy variable
    
    #compute the bitwise xor matrix
    M1 = bitxormatrix(genl1)
    M2 = np.triu(bitxormatrix(genl2),1) 

    for i in range(m-1):
        for j in range(i+1, m):
            [r,c] = np.where(M2 == M1[i,j])
            for k in range(len(r)):
                VT[(i)*n + r[k]] = 1;
                VT[(i)*n + c[k]] = 1;
                VT[(j)*n + r[k]] = 1;
                VT[(j)*n + c[k]] = 1;
                
                if M is None:
                    M = np.copy(VT)
                else:
                    M = np.concatenate((M, VT), 1)
                
                VT = np.zeros((n*m,1), int)
    
    return M
\end{minted}

При этом добавив к нему кучу красивого оформления.
